The goal of this library is to provide a C++ J\+S\+ON interface that\textquotesingle{}s simple to use and doesn\textquotesingle{}t require any dependencies. This library provides all necessary operations required for working with J\+S\+ON documents (see \href{#expamles_section}{\tt examples}).

{\bfseries Note\+:} the library only works with C++ 11 and newer.


\begin{DoxyItemize}
\item \href{#cpp_representation_section}{\tt C++ Representation of J\+S\+ON}
\item \href{#integration_section}{\tt Integration}
\item \href{#doc_section}{\tt Documentation}
\item \href{#tests_section}{\tt Tests}
\item \href{#expamles_section}{\tt Examples}
\begin{DoxyItemize}
\item \href{#parsing_json_string_section}{\tt Parsing a J\+S\+ON string}
\item \href{#opening_changing_section}{\tt Opening, changing and saving a J\+S\+ON file}
\item \href{#working_with_simple_section}{\tt Working with simple json\+\_\+variant values}
\item \href{#working_with_json_array_section}{\tt Working with json\+\_\+array}
\item \href{#working_with_json_object_section}{\tt Working with json\+\_\+object}
\item \href{#typecast_exceptions_section}{\tt Typecast exceptions}
\begin{DoxyItemize}
\item \href{#handling_typecast_expception_section}{\tt Handling typecast a exception}
\item \href{#avoid_typecast_exception_section}{\tt Avoid typecast a exception}
\end{DoxyItemize}
\item \href{#json_and_string_escapes_section}{\tt J\+S\+ON and string escapes}
\end{DoxyItemize}
\item \href{#customization_section}{\tt Customization}
\item \href{#troubleshooting_section}{\tt Troubleshooting}
\end{DoxyItemize}

\subsubsection*{\label{_cpp_representation_section}%
C++ Representation of J\+S\+ON}

All possible J\+S\+ON values are represenented in the class {\ttfamily json\+\_\+variant}. The datatypes are\+:
\begin{DoxyItemize}
\item int value\+: {\ttfamily json\+\_\+int} (typedef of {\ttfamily std\+::int64\+\_\+t})
\item double value\+: {\ttfamily double}
\item bool value\+: {\ttfamily bool}
\item string value\+: {\ttfamily std\+::string}
\item array value\+: {\ttfamily json\+\_\+array} (typedef of {\ttfamily std\+::vector$<$json\+\_\+variant$>$})
\item object value\+: {\ttfamily json\+\_\+object} (typedef of {\ttfamily std\+::map$<$std\+::string, json\+\_\+variant$>$})
\end{DoxyItemize}

\subsubsection*{\label{_integration_section}%
Integration}

Since the library consists only of the two files \mbox{[}\hyperlink{json__library_8h_source}{json\+\_\+library.\+h}\mbox{]}(G\+I\+T\+H\+U\+B\+\_\+\+U\+R\+L\+\_\+\+T\+O\+DO) and \mbox{[}json\+\_\+library.\+cpp\mbox{]}(G\+I\+T\+H\+U\+B\+\_\+\+U\+R\+L\+\_\+\+T\+O\+DO) you can just download them and add them into your project. To use the library, you should use {\ttfamily \#include \char`\"{}json\+\_\+library.\+h\char`\"{}} and define the namespace {\ttfamily using namespace json;} (recommended).

If you prefer to add the files seperated by classes you can download the \mbox{[}json\+\_\+library\mbox{]}(G\+I\+T\+H\+U\+B\+\_\+\+U\+R\+L\+\_\+\+T\+O\+DO) directory and include the directory into your project. To use the library in this case, you should use {\ttfamily \#include \char`\"{}json\+\_\+library\+\_\+source/json\+\_\+library.\+h\char`\"{}}.

\subsubsection*{\label{_doc_section}%
Documentation}

The library is documented with doxygen. The documentation can be found \mbox{[}here\mbox{]}(G\+I\+T\+H\+U\+B\+\_\+\+U\+R\+L\+\_\+\+T\+O\+DO).

\subsubsection*{\label{_tests_section}%
Tests}

The library was thoroughly tested. The test cases are written with the Qt Test Library and can be found \mbox{[}here\mbox{]}(G\+I\+T\+H\+U\+B\+\_\+\+U\+R\+L\+\_\+\+T\+O\+DO).

\subsubsection*{\label{_expamles_section}%
Examples}

The following examples will show how to work with the library.

\paragraph*{\label{_parsing_json_string_section}%
Parsing a J\+S\+ON string}


\begin{DoxyCode}
// include library
#include "json\_library.h"

// for convenience
using namespace json;

int main(int, char *[]) \{

    // initializing a string
    std::string string\_value = "\{ \(\backslash\)"name\(\backslash\)": \(\backslash\)"user\(\backslash\)", \(\backslash\)"age\(\backslash\)": 22, \(\backslash\)"favourite\_values\(\backslash\)": [2, null, false,
       \{\}] \}";

    // initializing a string with an easier writing - content same as above
    string\_value = R"(\{
        "name": "user",
        "age": 22,
        "favourite\_values": [2, null, false, \{\}]
    \})";

    // parse string
    json\_variant json = json\_variant::parse(string\_value);

    // print json into console
    std::cout << json << std::endl;             // prints the json with default indent of 4
    std::cout << json.dump(2) << std::endl;     // prints the json with indent of 2
    std::cout << json.dump(0) << std::endl;     // prints the json no indent and no new lines:
       "\{"age":22,"favourite\_values":[2,null,false,\{\}],"name":"user"\}"

    std::cout << json.is\_valid\_json() << std::endl; // prints "1" because the source string is valid json
       and was successfully parsed
    std::cout << json.is\_array() << std::endl;      // prints "0" because the root is an object and not an
       array

    if (json.is\_object()) \{
        std::cout << json.to\_object()["name"].to\_string() << std::endl;                 // prints "user"

        json\_object &obj = json.to\_object();
        std::cout << obj.at("favourite\_values").to\_array()[2].is\_bool() << std::endl;   // prints "1" since
       the json array holds a "false" bool value in index 2

        std::cout << obj.at("favourite\_values").dump(0) << std::endl;                   // prints
       "[2,null,false,\{\}]"
    \}

    return 0;
\}
\end{DoxyCode}


\paragraph*{\label{_opening_changing_section}%
Opening, changing and saving a J\+S\+ON file}


\begin{DoxyCode}
// include library
#include "json\_library.h"

// for convenience
using namespace json;

int main(int, char *[]) \{
    /*
    example assumes "values.json" is a text file that contains the following json object:
    \{
        "name": "user",
        "age": 22,
        "favourite\_values": [2, null, false, \{\}]
    \}
    */
    json\_document doc("values.json");

    if (doc.open()) \{ // only successfull if document contains valid json syntax and the file was found
        std::cout << doc.json() << std::endl;
        /*
        prints:
        \{
            "age": 22,
            "favourite\_values": [
                2,
                null,
                false,
                \{

                \}
            ],
            "name": "user"
        \}
        */

        // add a new subvalue in "favourite\_values" and change the "age"
        json\_variant &var = doc.json();         // the reference symbol is important if you want to change
       the document source
        json\_object &obj = var.to\_object();
        obj.at("age") = 44;
        json\_array &arr = obj["favourite\_values"].to\_array();
        json\_object &obj2 = arr[3].to\_object();
        obj2.insert(\{"favourite sub value", -11\});

        std::cout << doc.json() << std::endl;
        /*
        prints:
        \{
            "age": 44,
            "favourite\_values": [
                2,
                null,
                false,
                \{
                    "favourite sub value": -11
                \}
            ],
            "name": "user"
        \}
        */

        if (doc.save()) \{ // only successfull if document only contains valid json\_variants (variants with
       values) and the filename is valid
            std::cout << "document saved!" << std::endl;

            doc.close(); // optional
        \}
    \}

    return 0;
\}
\end{DoxyCode}


\paragraph*{\label{_working_with_simple_section}%
Working with simple json\+\_\+variant values}


\begin{DoxyCode}
json\_variant var;           // the json variant currently holds no value and is invalid

std::cout << var.is\_invalid\_json() << std::endl;    // will print "1"

var = json\_variant::null;   // setting the json\_variant to null
// NOTE: to initalize a json\_variant with null you have to use json\_variant::null, using for expample
       nullptr will NOT WORK.

std::cout << var.is\_invalid\_json() << std::endl;    // will print "0"
std::cout << var.is\_null() << std::endl;            // will print "1"

var = 234;                  // setting the json\_variant to json\_int

var = 0.345;                // setting the json\_variant to double

var = 2E-2;                 // setting the json\_variant to double in scientific notation

var = true;                 // setting the json\_variant to bool

var = "string value";       // setting the json\_variant to std::string
\end{DoxyCode}


\paragraph*{\label{_working_with_json_array_section}%
Working with json\+\_\+array}

Keep in mind that a {\ttfamily json\+\_\+array} just a {\ttfamily typedef} for {\ttfamily std\+::vector$<$json\+\_\+variant$>$} so all {\ttfamily std\+::vector} functions work on {\ttfamily json\+\_\+array} as well.


\begin{DoxyCode}
json\_variant var;                               // the json variant currently holds no value and is invalid

var = json\_array();                             // setting the json\_variant to json\_array
var = json\_array(\{234, 0.345, "string value"\}); // the json\_array holds json\_variants so this is valid
       syntax

// add array value
var.to\_array().push\_back("new value");          // add json\_variant element "new value"
var.to\_array().push\_back(false);                // adds json\_variant element "false"

var.to\_array()[2] = "string value 2";           // overwrites "string value" with "string value 2"

// using a array reference
json\_array &arr = var.to\_array();               // The reference symbol is important, otherwise you get a
       copy and won't modify the source array
arr.push\_back(456);                             // add json\_variant element "456"
arr[0] = 567;                                   // overwrites "234" with "567"
\end{DoxyCode}


\paragraph*{\label{_working_with_json_object_section}%
Working with json\+\_\+object}

Keep in mind that a {\ttfamily json\+\_\+object} just a {\ttfamily typedef} for {\ttfamily std\+::map$<$std\+::string, json\+\_\+variant$>$} so all {\ttfamily std\+::map} functions work on {\ttfamily json\+\_\+array} as well.


\begin{DoxyCode}
json\_variant var;                               // the json variant currently holds no value and is invalid

var = json\_object();                                                                // initalizing with
       empty json object
var = json\_object(\{ \{"property 1", true \}, \{ "property 2", json\_array(\{1, 2\}) \} \}); // using the c++ 11 way
       to initialize

// adding new properties with std::pair
var.to\_object().insert(std::pair<std::string, json\_variant>("new property 3", -2 ));

std::cout << var.to\_object().at("new property 3").to\_int() << std::endl; // prints "-2"

// adding new properties with c++ 11
var.to\_object().insert(\{"new property", "new value"\});
var.to\_object().insert(\{"new property 2", json\_variant::null \});

// changing a value
var.to\_object().at("property 1") = false;

std::cout << var.to\_object().at("property 1").to\_bool() << std::endl; // prints "0" (false)

// using a object reference
json\_object &obj = var.to\_object(); // The reference symbol is important, otherwise you get a copy and
       won't modify the source object
obj.at("property 2") = 100;
obj.insert(\{"property 5", -100\});
\end{DoxyCode}


\paragraph*{\label{_typecast_exceptions_section}%
Typecast exceptions}

Everytime a {\ttfamily json\+\_\+variant\+::to\+\_\+$\ast$type$\ast$()} ({\ttfamily to\+\_\+int()}, {\ttfamily to\+\_\+string()}, ...) gets called the json\+\_\+variant will try to cast the underlying value to the requested value. If the actuall value isn\textquotesingle{}t the requested value a {\ttfamily std\+::runtime\+\_\+error} will be thrown. To avoid handling exceptions you have to check whether the underlying value is actually the requested type. You can do so by using {\ttfamily is\+\_\+$\ast$type$\ast$()} ({\ttfamily is\+\_\+int()}, {\ttfamily is\+\_\+string()}, ...)

\subparagraph*{\label{_handling_typecast_expception_section}%
Handling typecast a exception}


\begin{DoxyCode}
json\_variant var = 234;

try \{
    // won't work since var is int
    double d = var.to\_double();  
    // ...
\}
catch (const std::runtime\_error &e) \{
    std::cout << "runtime\_error: " << e.what() << std::endl;   
\} 
\end{DoxyCode}


\subparagraph*{\label{_avoid_typecast_exception_section}%
Avoid typecast a exception}


\begin{DoxyCode}
json\_variant var = 234;

// returns false
if (var.is\_double()) \{
    double d = var.to\_double();   
    // ...
\}
else \{
    std::cout << "var is not double" << std::endl;   
\}
\end{DoxyCode}


\paragraph*{\label{_json_and_string_escapes_section}%
J\+S\+ON and string escapes}

All strings in an {\ttfamily json\+\_\+variant} and its children will be unicode escaped before being converted to a string (for example when calling {\ttfamily json\+\_\+variant\+::dump()}). However only the characters " and \textbackslash{} will be escaped since those are the only illegal characters in an J\+S\+ON string. json\+\_\+variant\+::unescape\+\_\+string() is able unescape A\+S\+C\+I\+I-\/characters (other escaped characters will stay escaped).


\begin{DoxyCode}
json\_variant var = R"(raw json string \(\backslash\)u3093with \(\backslash\)" very \(\backslash\) illegal / \(\backslash\)\(\backslash\) \(\backslash\)t "" characters")";

std::string escaped = R"("raw json string \(\backslash\)u005cu3093with \(\backslash\)u005c\(\backslash\)u0022 very \(\backslash\)u005c illegal / \(\backslash\)u005c\(\backslash\)u005c
       \(\backslash\)u005ct \(\backslash\)u0022\(\backslash\)u0022 characters\(\backslash\)u0022")";

std::cout << var << std::endl;
// prints: "raw json string \(\backslash\)u005cu3093with \(\backslash\)u005c\(\backslash\)u0022 very \(\backslash\)u005c illegal / \(\backslash\)u005c\(\backslash\)u005c \(\backslash\)u005ct
       \(\backslash\)u0022\(\backslash\)u0022 characters\(\backslash\)u0022"

std::cout << json\_variant::unescape\_string(escaped) << std::endl;
// pritns: "raw json string \(\backslash\)u3093with \(\backslash\)" very \(\backslash\) illegal / \(\backslash\)\(\backslash\) \(\backslash\)t "" characters""
\end{DoxyCode}


\subsubsection*{\label{_customization_section}%
Customization}

The behaviour of the library can be changed by changing defines in the header. The header will be either called \mbox{[}\hyperlink{json__library_8h_source}{json\+\_\+library.\+h}\mbox{]}(G\+I\+T\+H\+U\+B\+\_\+\+U\+R\+L\+\_\+\+T\+O\+DO) or \mbox{[}json\+\_\+library/json\+\_\+library.\+h\mbox{]}(G\+I\+T\+H\+U\+B\+\_\+\+U\+R\+L\+\_\+\+T\+O\+DO) if you\textquotesingle{}ve added the complete directory.


\begin{DoxyItemize}
\item Defining {\ttfamily J\+S\+O\+N\+L\+I\+B\+\_\+\+V\+E\+R\+B\+O\+S\+E\+\_\+\+D\+E\+B\+UG} will enable writing debug messagess into {\ttfamily std\+::cout} and {\ttfamily std\+::cerr} that may be helpfull to you but those messages mostly report internal library errors.
\item Changing the datatype in {\ttfamily J\+S\+O\+N\+L\+I\+B\+\_\+\+I\+N\+T\+\_\+\+T\+Y\+PE} is possible and may be helpfull if you for expample already know that youre only working with short ints and want to reduce the memory usage. The datatype must be a {\itshape signed} int.
\end{DoxyItemize}

\subsubsection*{\label{_troubleshooting_section}%
Troubleshooting}


\begin{DoxyItemize}
\item The libraries standard namespace is called {\ttfamily json}. If there are conflicting naming issues in your project you can change the namespace by changing the define in {\ttfamily J\+S\+O\+N\+L\+I\+B\+\_\+\+N\+A\+M\+E\+S\+P\+A\+CE} in the header (where to find the header is explained in \href{#customization_section}{\tt Customization}). 
\end{DoxyItemize}